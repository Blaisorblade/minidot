% ----- listings

\lstdefinelanguage{Scala}%
{morekeywords={abstract,%
  case,catch,char,class,%
  def,else,extends,final,finally,for,%
  if,import,implicit,%
  match,module,%
  new,null,%
  object,override,%
  package,private,protected,public,%
  for,public,return,super,sealed,%
  this,throw,trait,try,type,%
  val,var,%
  with,while,%
  yield%
  },%
  sensitive,%
  morecomment=[l]//,%
  morecomment=[s]{/*}{*/},%
  morestring=[b]",%
  morestring=[b]',%
  showstringspaces=false%
}[keywords,comments,strings]%

\lstset{language=Scala,%
  mathescape=true,%
%  columns=[c]fixed,%
%  basewidth={0.5em, 0.40em},%
  aboveskip=5pt,%\smallskipamount,
  belowskip=5pt,%\negsmallskipamount,
  lineskip=-0.2pt,
  basewidth={0.54em, 0.4em},%
%  backgroundcolor=\color{listingbg},
  basicstyle=\small\ttfamily,
  keywordstyle=\keywordstyle
%  commentstyle=\commentstyle
%  xleftmargin=0.5cm
}

\definecolor{listingbg}{RGB}{240, 240, 240}

\newcommand{\commentstyle}[1]{\slseries{#1}}
\newcommand{\keywordstyle}[1]{\bfseries{#1}}

\lstnewenvironment{listing}{\lstset{language=Scala,mathescape=true}}{}
\lstnewenvironment{listingtiny}{\lstset{language=Scala,mathescape=truebasicstyle=\scriptsize\ttfamily}}{}

\newcommand{\code}[1]{\lstinline[language=Scala,mathescape=true,columns=fixed,basicstyle=\ttfamily]|#1|}

% ----- Common inline listings (keywords, library methods etc.)
\newcommand{\lval}{\code{val}}
\newcommand{\ldef}{\code{def}}
\newcommand{\lfor}{\code{for}}
\newcommand{\lRand}[1][]{\code{Rand#1}}
\newcommand{\lflatMap}[1][]{\code{flatMap#1}}
\newcommand{\lchoice}[1][]{\code{choice#1}}
\newcommand{\lalways}[1][]{\code{always#1}}
\newcommand{\lnever}{\code{never}}
\newcommand{\lflip}[1][]{\code{flip#1}}
\newcommand{\lif}{\code{if}}
\newcommand{\lopand}{\code{\&\&}}
\newcommand{\lopor}{\code{||}}
\newcommand{\lopeq}{\code{==}}
\newcommand{\lplus}{\code{+}}
\newcommand{\lminus}{\code{-}}

% ----- packed items, so we don't waste space
\newenvironment{sitemize}{
\begin{itemize}
  \setlength{\itemsep}{1pt}
  \setlength{\parskip}{0pt}
  \setlength{\parsep}{0pt}
}{\end{itemize}}

\newenvironment{senumerate}{
\begin{enumerate}
  \setlength{\itemsep}{1pt}
  \setlength{\parskip}{0pt}
  \setlength{\parsep}{0pt}
}{\end{enumerate}}

\newcommand{\mypar}[1]{{\bf #1.}}

% ----- comments and todo

\newcommand{\remark}[1]{{\bf $\clubsuit$ #1 $\spadesuit$}}
%\newcommand{\note}[1]{\remark{\color{red}[#1]}}
\newcommand{\note}[1]{{\color{red}[#1]}}
\newcommand{\todo}[1]{\note{TODO: #1}}

\newcommand{\comment}[1]{}
